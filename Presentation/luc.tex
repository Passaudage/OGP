% Marketing partie 2
\begin{frame}{Quantité de distribution}
\begin{itemize}
\item<1-> Approvisionnement initial des distributeurs avec une quantité dépendant de leur rayonnement
\item<2-> Gestion des stocks de pièces de bases en kanban par les distributeurs
\item<3-> Contractualisation éventuelle d'un stock minimal présent chez le distributeur
\end{itemize}
\end{frame}

\begin{frame}{Localisation de la distribution}
	\begin{itemize}
		\item<1-> Format du meuble adapté à la distribution en grande surface (spécialisée en bricolage ou de grande distribution)
		\item<2-> Cible géographique : implantation initiale sur les marchés savoyards (localité) et parisiens (visibilité)
		\item<3-> Expansion progressive sur tout le marché français prévue sur 8 mois.
	\end{itemize}
\end{frame}

\begin{frame}{Prévisions de vente}
	\begin{table}
	\centering
	\begin{tabular}{>{\bfseries}r>{}m{0.23\textwidth}cc}
		\toprule
		& \textbf{Pied de table (x8)} & \textbf{Blocs standards (x24)} &
		\textbf{Blocs
		thermiques (x6)} \\
		\midrule
		Juin 16 & 30 & 27 & 12\\
		Juil. 16 & 90 & 81 & 36\\
		Août 16 & 210 & 219 & 88\\
		Sept.16 & 432 & 465 & 198 \\
		Oct. 16 & 765 & 800 & 402\\
		Nov. 16 & 1024 & 1200 & 580 \\
		Déc. 16 & 1148 & 1345 & 700 \\
		Janv. 17 & 1224 & 1445 & 732\\
		Fév. 17 & 900 & 1600 & 700\\
		Mars 17 & 800 & 1585 & 700\\
		Avril 17 & 750 & 1615 & 680\\
		Mai 17 & 800 & 1579 & 700\\
		\bottomrule
	\end{tabular}
	\caption{Prévisions de vente pour la période Juin 2016 -- Mai
	2017\label{prevVente}}
\end{table}
\end{frame}