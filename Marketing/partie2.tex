\section{Quantité de distribution}
L'un des arguments marketing principaux de la nouvelle table 
produite par PICASO est sa très grande modularité. Il est donc
nécessaire de mettre en \oe uvre une politique de distribution qui
permette au client de bénéficier de cet atout majeur. Nous envisageons
donc une politique de distribution autorisant le client final à choisir
finement les pièces qu'il souhaite acquérir. Nous proposons donc de
fournir les pièces de base packagées avec les pièces annexes servant à
l'assemblage disponible à l'acquisition individuelle (par exemple, un
bloc est vendu avec quatres taquets). Les différents distributeurs 
sont approvisionnés par palette, une étude de marché permettant de
prévoir au mieux les exigences des clients.

\section{Localisation de la distribution}
Afin de toucher un public large et varié, la distribution se fait chez des
marchands généralistes (types hypermarché) où dans des grandes surfaces
plus spécialisées dans l'ameublement et le bricolage (type Leroy Merlin,
ou Castorama). Nous n'envisageons pas pour le moment d'attaquer un
marché international. Il sera peut-être envisageable de le faire dans un
avenir proche si le concept marche bien en France. Il est à noter que
nous ciblerons en priorité les région Rhônes-Alpes Auvergnes et Île de
France. La première pour une raison de proximité et la seconde bour
bénéficier d'une aubaine publicitaire intéressante (c'est en effet là
que sont regroupés tous les journalistes des médias à diffusion
nationale).

\section{Politique de gestion de la production}
Afin de permettre aux clients finaux de bénéficier de l'intérêt que la
table apporte par sa modularité, il est nécessaire que les pièces des
variantes principales soient facilement accessibles pour que ces
derniers puissent par exemple acheter facilement et rapidement une
rallonge pour leur table.

Une étude de marché visera à établir les couleurs les plus recherchées,
ce qui permettra de fournir un stock initial adéquat aux distributeurs.
Ensuite, ceux-ci géreront en kanban l'approvisionnement des différentes
variantes. Il sera envisageable, si la marque prend de la notoriété de
faire signer un contrat d'engagement de stock minimal chez les
distributeurs.
\section{Localisation de la fabrication}
La production restera, tant que le site industriel le permettra,
localisée dans l'usine historique, située en Savoie. Cela permettra de
gérer de façon centralisée les approvisionnements et ainsi négocier de
meilleurs prix. De plus, cela permet de donner une identité forte à la
marque, ce qui lui permettra de bénéficier d'un à priori favorable de la
cible commercial.

\section{Prévisions de vente}
Tous les nouveaux produits suivent initialement la même tendance lors du
lancement du produit : pendant la phase de mise sur le marché, les lots
de bases seront fournis à un nombre toujours croissants de
distributeurs. Ensuite, ces distributeurs entreront en gestion kanban
des modules, et la demande en modules basique (sans trous ni traitement
thermiques) sera certainement plus importante que celles en modules
thermiques.

En terme de prévision de vente, nous estimons que la phase de mise sur le
marché durera environ 8 mois. Nous donnons ensuite à titre indicatif les
prévisions de vente qui se stabilisent sur 4 mois. D'autre part, les
prévisions par couleurs ne sont pas détaillées puisque chaque couleur
représente une proportion non fluctuante des prévisions de vente. La
répartition des couleurs de base prévisionnelle est la suivante (basée
sur des chiffres publiés par un concurrent adaptés aux spécifités des
produits PICASO) :

\begin{itemize}
\item Bois naturel : 32 \%,
\item Blanc : 26\%,
\item Noir : 18\%,
\item Rouge : 8\%,
\item Jaune : 7\%,
\item Bleu : 7\%,
\item Vert : 2\%
\end{itemize}

Les prévisions de vente sont données en table \ref{prevVente}.
\begin{table}[!ht]
	\centering
	\begin{tabular}{>{\bfseries}rccc}
		\toprule
		& \textbf{Pied de table (x8)} & \textbf{Blocs standards (x24)} &
		\textbf{Blocs
		thermiques (x6)} \\
		\midrule
		Juin 16 & \\
		Juil. 16 & \\
		Août 16 & \\
		Sept.16 & \\
		Oct. 16 &\\
		Nov. 16 & \\
		Déc. 16 & \\
		Janv. 17 & \\
		Fév. 17 & \\
		Mars 17 & \\
		Avril 17 & \\
		Mai 17 &\\
		\bottomrule
	\end{tabular}
	\caption{Prévisions de vente pour la période Juin 2016 -- Mai
	2017\label{prevVente}}
\end{table}

Enfin, il faut noter que les prévisions de ventes des anciens produits
ne subissent pas de conséquences de la mise sur le maché de produits qui
ne les concurrencent pas.
