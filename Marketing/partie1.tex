\section{Description du produit}
Le produit que nous avons retenu est issu d'une intense réflexion sur les besoins de la clientèle. En effet, il est important de coller au plus près du besoin client et de ses attentes. 
De ce fait, nous avons retenu le concept de table en bloc. Le concept est simpliste mais personnalisable et moderne. Le principe est de décomposer la table, et notamment sont plateau, en plusieurs blocs de même forme, mais personnalisable de différentes façons.
A l'instar d'un échiquier, le client pourra agencer le plateau de sa table à souhait. Il pourra également disposer les pieds où bon lui semble, tout en faisant preuve de bon sens. 
Les blocs seront tous des carrés de 300mm de côté sur 50mm d'épaisseur. Cependant, il y aura plusieurs types de blocs, tous conçu sur la base d'un bloc standard. Voici une liste exhaustive des types de blocs :
\begin{itemize}
\item Bloc standard : Bloc carré de 300mm de côtés sur 50mm d'épaisseur. La couleur est personnalisable dans la palette des couleurs disponibles. Il possède des éléments de jonctions avec d'autres blocs. 
\item Bloc thermique : Bloc carré de 300mm de côtés sur 50mm d'épaisseur. La couleur est personnalisable dans la palette des couleurs disponibles. Il possède des éléments de jonctions avec d'autres blocs. En plus d'un bloc standard, il possède un traitement thermique qui permet de poser sur ce bloc un appareil de cuisson, comme un appareil à fondue ou à raclette.
\item Bloc électrique : Bloc carré de 300mm de côtés sur 50mm d'épaisseur. La couleur est personnalisable dans la palette des couleurs disponibles. Il possède des éléments de jonctions avec d'autres blocs. En plus d'un bloc standard, il possède un trou au centre afin de faire passer un câble électrique.
\item Bloc électrique thermique : Il est bien sur possible d'effectuer sur un bloc le traitement thermique et de monter le passe câble électrique. 
\end{itemize}

\section{Nature d'utilisation}
Ce mobilier pourra être mise en place pour répondre à plusieurs cas d'usage. On pourra l'utiliser en tant que bureau si l'on utilise des blocs standards et des blocs électriques. Il pourra être implanté dans une cuisine si on prend soin de faire le traitement thermique sur tous les blocs. Sa modularité et son potentiel de personnalisation fait de ce produit un incontournable pour toutes les personnes qui souhaitent une table polyvalente.

\section{Degré de qualité}
Notre produit se situera en milieu de gamme. Nous ne viserons pas une qualité équivalente aux standards de l'industrie du meuble de luxe. Cependant nous prendrons soin d'offrir une qualité satisfaisante à notre clientèle, tout en respectant la contrainte d'un prix minimum. La qualité des matériaux et donc des produits finaux se situera de le moyen de gamme.

\section{Nomenclature de planification}
%% TODO

\section{Nature des matériaux utilisés}
Les blocs ainsi que les pieds seront fabriqués en sapin, provenant de scieries locales. Nous achèterons les blocs et les pieds bruts. Ceux ci seront ensuite usinés dans notre usine pour les transformer en produits intermédiaires, utilisés par la suite dans le processus de différenciation retardée pour la personnalisation du produit final. 
Les taquets, les chevilles et les caches seront achetés à nos fournisseurs car ce sont des produits standards. 

\section{Nature de la distribution}
Nous aurons plusieurs canaux de vente, qui permettront de toucher une plus large clientèle.
Nos produits seront vendus en grande surface. Ce mode de distribution permet de toucher une partie de la clientèle plutôt âge qui est encore réfractaire au mode de distribution moderne comme internet. Nos produits seront mis à disposition du client dans de grandes enseignes d'ameublement ainsi que dans des magasins plus spécialisés. 
Nos produits seront également disponible en vente par correspondance, sur internet. Nous mettrons en place un configurateur qui permettra de créer son modèle de table en ligne de manière interactive. Ainsi le client pourra avoir un rendu visuel de sa table et également un calcul automatique du prix. Il pourra donc innover et tester des arrangements de blocs selon ses envies et ses contraintes. 

\section{Délai de mise à disposition}
Le délai de mise à disposition dépend de la nature du mode de distribution.
En effet, si le client achète son produit en magasin, dans l'hypothèse ou le magasin possède un stock suffisant, le délai de mise à disposition est immédiat. Si le stock n'est pas suffisant, le délai de réapprovisionnement doit être de 3 jours ouvrés maximum afin que le client attende le moins possible.
Si le client a passé commande sur internet, nous devons nous engager à le livrer au maximum 7 jours ouvrés après réception du paiement. Cette durée est un standard dans les sites de e-commerce pour un service standard. Le client pourra également souscrire un service premium qui lui permettra de recevoir son mobilier dans un délai de 3 jours ouvrés. 