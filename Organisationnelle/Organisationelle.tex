\documentclass[a4paper]{TPInsa}

\title{Dossier bilan}

\begin{document}
	\pageTitre
	\tableMatieres

		L'objectif de la société PICASO est de diversifier sa gamme de produits de façon à mieux répondre à la demande de ses clients. Nous devons aussi guider la mise en place d'un processus de gestion de production de type MRP2.	
	
	\section{Situation actuelle}
	
	\subsection{Description de l'entreprise}
	
	La société PICASO est une entreprise industrielle spécialisée dans le mobilier de salon pour les particuliers et les entreprises. La maison mère est installée à Lons-Le-Saunier. Les produits sont principalement vendus en grandes surfaces généralistes, en magasins d'ameublement et auprès des fournisseurs de mobilier de bureau. Ses savoir-faire métiers sont la découpe, le travail et l'assemblage du bois.
	
	Sur le site d'Ayze en Haute-Savoie, la société Picaso a décidé de se concentrée sur les bibliothèques basses en pin. Ce produit est très simple, intemporel et bon marché. La gamme des bibliothèques basses est composée des modèles ARM100 et ARM200, bibliothèques de respectivement 100cm et 200cm.  
	
	\fixedWidthFigure{bibliotheque_basse.jpg}{ARM100}{0.6\textwidth}
	
	L'entreprise a également tissé des relations privilégiées avec des fournisseurs en qui l'ont peu avoir une confiance aveugle. 
	
	\subsection{Description du personnel}
	L'entreprise fonctionne en 1/8 : 5 jours par semaine, 8 heures par jours. Une seule équipe de production est présente 8 heures par jour sur le site d'Ayze. Cela équivaut a un total de 40 heures par personne à la semaine. 
	
	Le personnel est composé de ... %TODO
	\subsection{Description du patrimoine matériel}
	L'usine d'Ayze est composé de plusieurs machines permettant la fabrication et l'assemblage des bibliothèques basses ARM100 et ARM200. 
	La parc de machines est le suivant :
	\begin{itemize}
	\item 3 machines de découpes : DEC1, DEC2, DEC3
	\item 3 machines de traitement du bois : MB1, MB2, MB3
	\item 1 machine d'assemblage de sous ensemble : LASE
	\item 2 machines d'assemblage final : LAF1, LAF2
	\end{itemize}
	
	Le mode de fonctionnement est actuellement du 1/8 mais il n'y aucune restriction à un fonctionnement en 2/8 (c'est à dire 16h de travail par jour).
	\section{Situation future}
	
\end{document}
