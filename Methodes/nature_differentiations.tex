\section{Nature des différentiations}

Le produit final est constitué d'un ensemble d'articles appelés blocs.
Chacun de ces blocs est issu de la différentiation d'un bloc standard,
ce qui constitue le principe de la différentiation retardée proposée ici.
Il s'agit ainsi de proposer trois opérations permettant d'offrir une
spécialisation fonctionnelle de chaque bloc.
\begin{itemize}
	\item Une peinture peut être appliquée sur le bloc
	dont la couleur peut être choisie parmi un ensemble de teintes proposées,
	cependant le client a toujours la possibilité de demander une couleur
	personnalisée. Cette couleur offre une personnalisation de la table finale
	qui s'adapte alors à l'intérieur du client.
	
	\item Lorsqu'un bloc standard n'est pas revêtu d'une peinture ou d'un
	traitement thermique spécifique, il est nécessaire d'appliquer un
	traitement de surface sous forme de lasure,
	permettant la protection du bois.
	
	\item Un traitement thermique appliqué sur la surface du bois permet
	d'offrir à un bloc standard une protection en vue de l'utilisation
	de matériels chauffants sur le bloc. Il est ainsi possible de poser
	des appareils du type \og appareil à raclette \fg{}, ou même des plats
	chauds directement sur la table sans craindre d'abîmer le bloc.
	
	\item Un perçage dans le bloc standard rend compatible l'utilisation
	d'appareils divers munis d'un câble électrique, celui-ci pouvant
	plonger au milieu de la table sans gêner les utilisateurs.
	Ce passe-câble est de diamètre 5cm de manière à faire passer tout type
	de connecteurs électriques existants.
	
\end{itemize}

De même, les pieds de la table se voient proposer deux différentiations
possibles,
\begin{itemize}
	\item une peinture aux teintes identiques à celles proposées 
	pour les blocs,
	\item ainsi qu'une lasure de protection pour le bois, correspondant
	également à celle utilisée pour les blocs lorsqu'aucune peinture
	n'est appliquée. 
\end{itemize}

L'ensemble de ces trois différentiations retardées sont compatibles
entre elles. Un même bloc peut donc proposer des combinaisons de
différentiations, tel que décrit dans le dossier marketing.
