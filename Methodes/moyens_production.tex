\section{Moyens de production}

Le produit ainsi proposé nécessite la mise en place de nouveaux
moyens de production et postes de travail. En effet, bien que de nombreux
postes de travail spécifiques au bois sont présents -- poste de perçage,
débit et découpe notamment --, certaines opérations impliquent une adaptation
de l'entreprise.

Afin de mener à bien les opérations de conditionnement des produits,
il est nécessaire à l'entreprise de se doter de deux machines d'emballage.
L'une permet la mise en sachet de petite et moyenne taille de tout type
d'objet, et l'autre se prête au conditionnement des articles blocs et pieds
respectivement en les recouvrant d'un film plastique.
Deux opérateurs sur ces machines doivent alors procéder à leur alimentation
en fonction des commandes clients.

Une mélangeuse industrielle est requise pour la fabrication des couleurs
Pantone\textregistered{} à partir des peintures de teintes primaires.
Un opérateur spécifiquement formé à cette tâche est affecté à ce poste,
lorsque des demandes sont formulées.

La mise en place d'un poste de travail d'une taille conséquente, surmontée
d'une hotte doit permettre l'application des différents traitements de
surface sur le bois, comme par exemple les peintures, lasures et revêtement
contre la chaleur. La hotte assure la sûreté du poste du travail en évitant
aux opérateurs de respirer des produits potentiellement toxiques lors de
leur application et du séchage des pièces, en s'accordant avec la politique
sécurité déjà mise en place via le \textsc{Chsct} de l'entreprise.
En fonction des commandes de l'entreprise, un nombre d'employés assez
important est attendu à ce poste.
