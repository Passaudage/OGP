\documentclass[a4paper]{TPInsa}

\title{Dossier méthode}

\begin{document}
	\pageTitre
	\tableMatieres

	\section{Objectifs}

Ce dossier s'attache à présenter les différents éléments spécifiques à la
gestion de la production du site d'Ayze, dans le cadre de l'ajout dans la
gamme de produit de l'entreprise Picaso de l'article E-kitable.
De plus amples informations sur cet article peuvent être trouvées dans
le dossier Marketing ci-joint.


	\section{Nature des différentiations}

Le produit final est constitué d'un ensemble d'articles appelés blocs.
Chacun de ces blocs est issu de la différentiation d'un bloc standard,
ce qui constitue le principe de la différentiation retardée proposée ici.
Il s'agit ainsi de proposer trois opérations permettant d'offrir une
spécialisation fonctionnelle de chaque bloc.
\begin{itemize}
	\item Une peinture peut être appliquée sur le bloc
	dont la couleur peut être choisie parmi un ensemble de teintes proposées,
	cependant le client a toujours la possibilité de demander une couleur
	personnalisée. Cette couleur offre une personnalisation de la table finale
	qui s'adapte alors à l'intérieur du client.
	
	\item Lorsqu'un bloc standard n'est pas revêtu d'une peinture ou d'un
	traitement thermique spécifique, il est nécessaire d'appliquer un
	traitement de surface sous forme de lasure,
	permettant la protection du bois.
	
	\item Un traitement thermique appliqué sur la surface du bois permet
	d'offrir à un bloc standard une protection en vue de l'utilisation
	de matériels chauffants sur le bloc. Il est ainsi possible de poser
	des appareils du type \og appareil à raclette \fg{}, ou même des plats
	chauds directement sur la table sans craindre d'abîmer le bloc.
	
	\item Un perçage dans le bloc standard rend compatible l'utilisation
	d'appareils divers munis d'un câble électrique, celui-ci pouvant
	plonger au milieu de la table sans gêner les utilisateurs.
	Ce passe-câble est de diamètre 5cm de manière à faire passer tout type
	de connecteurs électriques existants.
	
\end{itemize}

De même, les pieds de la table se voient proposer deux différentiations
possibles,
\begin{itemize}
	\item une peinture aux teintes identiques à celles proposées 
	pour les blocs,
	\item ainsi qu'une lasure de protection pour le bois, correspondant
	également à celle utilisée pour les blocs lorsqu'aucune peinture
	n'est appliquée. 
\end{itemize}

L'ensemble de ces trois différentiations retardées sont compatibles
entre elles. Un même bloc peut donc proposer des combinaisons de
différentiations, tel que décrit dans le dossier marketing.

	\section{Dénominations}

\subsection{Articles achetés}

Les articles suivants sont achetés auprès de fournisseurs spécialisés
et constituent l'ensemble des éléments nécessaires à la fabrication des blocs,
des pieds de table ainsi que des accessoires de fixation :

\begin{itemize}
	\item sachets petite taille,
	\item sachets taille moyenne,
	\item film plastique de conditionnement,
	\item chevilles,
	\item caches plastique,
	\item planches de bois épaisseur 50mm,
	\item profilé de bois 70x70mm,
	\item lasure,
	\item peinture.
\end{itemize}

Différentes variantes de teinte de peinture sont disponibles :
noir, beige, blanc, gris, brun, rouge, bleu, vert.
Des nuances personnalisées Pantone\textregistered{} peuvent être proposées
pour la personnalisation, sur demande spécifique uniquement, et ceci grâce
au mélange des trois teintes primaires de peinture.

\subsection{Articles intermédiaires}

Suite aux différentes opérations de production et sur la base des articles
achetés, sont produits les articles intermédiaires suivants :

\begin{itemize}
	\item taquet 30x50x100mm, provenant du débit des planches de bois
	d'épaisseur 50mm à la bonne largeur et longueur,
	\item pieds 70x70x800m, ils sont débités à partir des profilés bois
	70x70mm, avec quatre perçages de manière à faire passer les chevilles
	fixant les pieds,
	\item blocs de dimension 300x300x50mm, découpés à partir des planches de
	bois d'épaisseur 50mm, avec les quatre perçages des taquets pour l'assemblage
	des blocs et les quatre perçages des chevilles de manière à fixer les pieds,
	ainsi que des caches plastiques s'insèrent dans les emplacements dédiés
	aux taquets lorsque ceux-ci sont inutilisés.
\end{itemize}

\subsection{Articles vendus}

L'entreprise propose un certain nombre d'articles vendus permettant d'offrir
l'ensemble des éléments nécessaires au montage de la table
\og personnalisé \fg{}.

\begin{itemize}
	\item chevilles en conditionnement de 4 unités à l'aide d'un sachet
	de petite taille,
	\item caches plastique en conditionnement de 4 unités à l'aide d'un sachet
	de petite taille,
	\item taquet 30x50x100mm, en conditionnement de 4 unités à l'aide
	d'un sachet de taille moyenne,
	\item bloc finalisé selon la nature de la différentiation,
	\item pied également finalisé selon la nature de la différentiation.
\end{itemize}

Les blocs envoyés au client sont groupés par x unités à l'aide d'un
film de conditionnement plastique.
De même, les pieds sont conditionnés de manière identique par groupe de x
unités.
Lorsque des blocs et des pieds possèdent respectivement des différentiations
identiques, ils sont préférentiellement conditionnés ensembles.

	\section{Articles et variantes}

Il existe deux articles standards pouvant bénéficier d'une différentiation
retardée, les blocs de base et les pieds de base.
Pour chacun de ces deux articles, il existe deux variantes correspondant
à l'application d'une lasure directement sur le bois ou bien d'une peinture
et d'un vernis, dont les teintes appartiennent à un ensemble de couleurs
prédéfinies.

Les blocs possèdent deux options spécifiques. L'une correspond au perçage
permettant le passage de câbles à travers les blocs, l'autre à l'application
d'un revêtement de type vernis permettant de protéger le bloc d'une forte 
source de chaleur. Ces deux options peuvent être simultanément présentes.


	
\end{document}


nature de la différenciation retardée : 
peinture,
traitement thermique
traitement bois nu (sans peinture) 
trou prise diamètre 50mm 


dénomination des articles achetés : 
taquet 30x50x100mm 
chevilles
caches plastique
planches de bois épaisseur 50mm 
planches de bois épaisseur 70mm 
peinture (noir, beige, blanc, gris, brun, rouge, bleu, vert, 

articles intermédiaires : 

pieds : 70*70*800mm découpés dans planches de bois épaisseur 70mm et surfacés, trou pour cheville 
bloques : 300*300*50mm découpés dans planches de bois épaisseur 50mm et surfacés, trous pour taquet (on met un cache s'il n'est pas utilisé) et trou pour cheville (pied) 



articles standards variantes et options : 
variantes : peintures,
options : traitement thermique 

nomenclatures


nouveaux moyens de prod nécessaires : emballage, peinture et traitement, "hote" pour éviter les vapeurs de bombes  


identification des méthodes d'approvisionnement de chacun des articles : 
caches plastique : sur conso 
approvisionnement de taquet 30x50x100mm : sur consommation 
peintures : colories courantes : sur consomation
colories moins courantes : sur besoin 

planches de bois epaisseur 50mm : sur conso 
chevilles : sur conso 

articles intermédiaires : 
