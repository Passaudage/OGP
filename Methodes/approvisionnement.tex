\section{Approvisionnement et planification}

Concernant l'ensemble des articles achetés, il est proposé de mettre en place
une politique d'approvisionnement sur consommation. De ce fait,
il est possible d'obtenir assez de stocks de manière à assurer la continuité
de la production s'il arrivait un incident d'approvisionnement.

Cependant et spécifiquement pour les teintes de peintures peu utilisées,
il est recommandé d'adopter une gestion sur besoin des stocks.
Ceci évite l'accumulation d'articles non utilisés dans ce dernier, ce qui
représente une perte d'espace pour les autres actifs circulant.

La planification de la production repose sur l'utilisation de la méthode
MRP1. Il s'agit de prendre en compte les données du plan industriel et
commercial et du programme de production pour adapter notamment 
l'organisation des différentes opérations de différentiations.

La fabrication des articles intermédiaires standardisés blocs et pieds, ainsi
que le débit des taquets met en jeu une production en ligne. De cette manière,
il est possible de produire régulièrement de grandes quantités puisque
les articles ne sont pas encore différenciés. L'approvisionnement
sur consommation permet une alimentation efficace des lignes de production.

Les opérations intervenant sur les articles intermédiaires, blocs standards et
pieds, sont ensuite réalisées dans le cadre d'une production par lots.
Cette organisation est stratégique pour permettre la fabrication en parallèle
d'un certain nombre d'articles différenciés selon les prévisions de
production et de ventes ainsi que les commandes personnalisées.
C'est la flexibilité qui est ici privilégiée et qui offre un niveau de
réactivité élevé pour la différentiation sur la base des articles standards.

Le plan stratégique de l'entreprise n'est pas pris en compte dans la politique
de planification globale, car les quantités d'articles proposés au public
ne tiennent pas lieu à des modifications à long terme. C'est pourquoi,
la méthode de gestion MRP1 est privilégiée à la méthode MRP2.

Il est à noter que les prévisions de production sont accessibles 
dans le dossier organisationnel et ne sont ainsi pas détaillées
dans ce document.
