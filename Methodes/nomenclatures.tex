\section{Nomenclatures}

Voici les nomenclatures correspondant aux blocs et aux pieds de table.

	\begin{figure}[H]
\centering
\includegraphics[scale=0.35]{./captures/Bloc_base.png}
\caption{Nomenclature du bloc}
	\end{figure}	

Le bloc standard correspond à une plaque de bois de 300*300*50 mm. Celui-ci a subi des transformations pour pouvoir accueillir les taquets ainsi qu'un pied. 
Chaque bloc est fourni avec 4 taquets. 
La différenciation retardée se fait à partir de là, il est possible d'appliquer de la peinture et du vernis ou bien de la lasure. On peut ensuite percer le bloc dans le but de faire passer un fil électrique, mais aussi lui appliquer un revêtement thermique pour qu'il résiste mieux à la chaleur. 
Dans le cas où il est percé, le bloc est fourni avec un cache.
	
		\begin{figure}[H]
\centering
\includegraphics[scale=0.35]{./captures/Pieds.png}
\caption{Nomenclature des pieds}
	\end{figure}	

La différenciation retardée se fait de la même manière pour les pieds. Il est seulement possible de le peindre ou de le lasurer, toujours d'une couleur choisie par le client.